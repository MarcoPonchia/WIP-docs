\section{Collaborazione}
\subsection{Comunicazioni}
\subsubsection{Comunicazioni esterne}
Per le comunicazioni esterne al gruppo sono stati previsti diversi canali.
Il principale tra questi è la \email\space \href{mailto:\GroupEmail}{\GroupEmail}\space che viene impiegata per le comunicazioni ufficiali.\\
Sotto suggerimento dei proponenti del progetto, viene utilizzato anche un canale Slack\texttrademark\footnote{\url{https://slack.com}}, per comunicazioni più dirette e rapide con gli stessi durante lo svolgimento del progetto. Nel caso in cui le comunicazioni scritte non dovessero essere sufficienti è stata prevista la possibilità di comunicare tramite Skype\texttrademark\footnote{\url{https://www.skype.com}} o incontrarsi in un luogo da concordarsi tra team di sviluppo e proponenti del progetto.

\subsubsection{Comunicazioni interne}
Il mezzo predefinito per le comunicazioni tra i membri del gruppo è un canale Slack\texttrademark.
In aggiunta a questo verrà utilizzato anche Discord\texttrademark\footnote{\url{https://discordapp.com/}} per le chat vocali.
Ogni comunicazione inerente a decisioni importanti di progetto tra membri del gruppo dovrà essere verbalizzata.
È stato previsto un incontro quotidiano tramite chat vocale della durata di 30 minuti, compatibilmente con gli impegni dei singoli, per tenersi aggiornati sui progressi e prendere le decisioni per cui si ritiene necessaria la presenza degli altri componenti. Per questi incontri deve essere selezionata una persona che si occupi di scrivere un verbale contenente le informazioni più importanti discusse durante l'incontro.

\subsubsection{Regole di stesura delle e-mail}
L'utilizzo della \email\space è previsto solamente per comunicazioni formali verso l'esterno, quindi il mittente dovrà essere necessariamente il nome del gruppo e la stessa \email\space inviata dal \textit{Responsabile di Progetto}.

\subsection{Riunioni}
Con il termine \virgolette{riunioni} sono incluse diverse occasioni di assemblea tra i membri del gruppo:
\begin{itemize}
\item Riunioni virtuali, nelle quali la comunicazione tra le persone avviene mediante un software di chat vocale. Questa modalità è da preferirsi in caso di difficoltà della maggior parte dei membri del gruppo a incontrarsi di persona e per comunicazioni più frequenti o di un numero ridotto di membri del gruppo.
\item Riunioni fisiche, in cui tutti i membri del gruppo si trovano in un concordato luogo.	
\end{itemize}
Le riunioni sono distinte in interne ed esterne in base alla partecipazione o meno dei proponenti e dei docenti.
Indipendentemente da questa divisione, ogni riunione deve essere verbalizzata. 
A rotazione vengono selezionate due persone incaricate di adempiere a questo obbligo di verbalizzazione.
Il documento deve essere titolato \textit{verbale}, seguito dalla tipologia della riunione (\textit{interna} o \textit{esterna}) e dalla data del giorno della stessa in forma breve, tutti separati da \_(underscore).\\
Il primo paragrafo deve descrivere l'ordine del giorno trattato nella riunione.\\

Per ciascuna delle tipologie di riunione vengono ora specificate:
\begin{itemize}
	\item frequenza delle riunioni;
	\item procedura di convocazione delle riunioni;
	\item svolgimento delle riunioni;
	\item modalità di verbalizzazione delle riunioni.	 
\end{itemize}

\subsubsection{Riunioni interne}
In questa sezione del documento vengono definite le caratteristiche delle riunioni interne, ossia quelle che includono esclusivamente i membri del gruppo.

\paragraph{Frequenza delle riunioni}\mbox{}\\
Per le riunioni interne è previsto che il gruppo si incontri una volta a settimana per riunioni tramite chat vocale e una volta ogni due settimane per riunioni di persona. Sono inoltre previste, in caso di necessità, riunioni interne di emergenza che vengono affrontate in \sezione \ref{sec:riunioni_emergenza}.

\paragraph{Convocazione delle riunioni}\mbox{}\\
Il compito di convocare le riunioni spetta al \textit{Responsabile di Progetto}, il quale deve provvedere a comunicarle tramite Teamwork\texttrademark\space agli altri membri del gruppo, con un preavviso di almeno quattro giorni. Il \textit{Responsabile di Progetto} deve altresì fornire la data, la durata dell'incontro e l'ordine del giorno, indipendentemente dalla tipologia di riunione. In caso di un incontro di persona deve anche rendere noto il luogo della riunione.\\
Gli altri membri devono comunicare la propria disponibilità tramite Teamwork\texttrademark, seguendo le procedure indicate in \sezione \ref{sec:procedure_riunioni}, e la riunione si terrà solamente se almeno quattro dei membri del gruppo sono disponibili. In caso contrario il \textit{Responsabile di Progetto} deve provvedere ad organizzare un'altra riunione, come descritto nella procedura.\\
Qualora fosse necessario un incontro tra un sottoinsieme dei membri del gruppo, per esempio in caso di necessità di confronto tra persone con funzioni affini o complementari, i membri sono liberi di organizzarsi tra loro, senza includere il \textit{Responsabile di Progetto}. Tutte le eventuali decisioni prese durante questo incontro devono poi essere presentate e discusse nella successiva riunione interna del gruppo e, se reputate valide, rese ufficiali e opportunamente verbalizzate.

\paragraph{Svolgimento delle riunioni}\label{sec:svolgimento_riunioni_interne} \mbox{}\\
L'andamento delle riunioni deve essere controllato dal \textit{Responsabile di Progetto}. Egli deve quindi preoccuparsi:

\begin{itemize}
	\item che vengano discussi tutti gli argomenti presenti nell'ordine del giorno;
	\item che i due incaricati prendano appunti per la successiva verbalizzazione della riunione;
	\item che la durata della riunione sia conforme con quanto precedentemente concordato;
	\item di controllare il generale flusso della discussione.
\end{itemize}
Vanno selezionate a rotazione tra i partecipanti della riunione due persone, le quali hanno l'incarico di mettere a verbale un resoconto delle argomentazioni più importanti dette durante la riunione, incluse le decisioni eventualmente prese. All'inizio della riunione vanno comunicati i progressi fatti da ciascuno e prese decisioni su come procedere con lo sviluppo del progetto. Segue quindi una trattazione degli argomenti specificati nell'ordine del giorno. 

\paragraph{Verbale delle riunioni}\label{sec:verbale_riunioni_interne}\mbox{}\\
Per ogni riunione va assegnato il compito di prendere appunti per la scrittura del verbale a due membri del gruppo. Dopo la riunione ad uno dei due viene assegnato dal \textit{Responsabile di Progetto} il compito di redigere il verbale, basandosi sugli appunti presi da entrambi. Dopo aver redatto il verbale la persona che lo ha scritto viene quindi sostituita da un'altra scelta a rotazione tra gli altri membri del gruppo, escluso il \textit{Responsabile di Progetto} in carica. Alla persona che la volta precedente aveva solo preso appunti e non redatto il verbale viene assegnato il compito di scriverlo, mentre il componente del gruppo appena entrato gli passerà gli appunti presi.\\
Avendo due persone con il compito di prendere appunti, la probabilità che qualche informazione importante venga persa viene ridotta, mentre lasciando il compito di scriverlo ad una sola si evitano eventuali problemi di organizzazione.
Questo inoltre permette a chi redige il verbale di aver già fatto esperienza con il prendere appunti ed aver visto un verbale già redatto.\\
Il documento prodotto verrà conservato sul repository \url{link repo}, condiviso tra tutti membri del progetto.

\paragraph{Riunioni interne d'emergenza}\label{sec:riunioni_emergenza}\mbox{}\\
È prevista lo possibilità di svolgere riunioni interne di emergenza in caso di necessità. Le riunioni di emergenza richiedono un anticipo di soltanto un giorno e seguono una procedura diversa da quella delle riunioni standard, descritta in \sezione \ref{sec:procedure_riunioni}. Inoltre, affinché una riunione di emergenza venga considerata ufficiale basta la presenza di tre membri del gruppo.\\
Tutte le decisioni prese durante una riunione di emergenza devono poi essere ridiscusse e analizzate durante la successiva riunione interna ordinaria.

\subsubsection{Riunioni esterne}
In questa sezione vengono affrontate le riunioni esterne, ovvero quelle che coinvolgono il committente, i proponenti o eventuali terze parti.

\paragraph{Frequenza delle riunioni}\mbox{}\\
La frequenza di questo tipo di riunioni non è fissata a priori, ma sarà valutata di volta in volta a seconda della necessità di confronto con i proponenti o con i committenti, relativamente allo svolgimento del progetto.

\paragraph{Convocazione delle riunioni}\mbox{}\\
Secondo quanto indicato da \Proponente\texttrademark, per comunicare con i proponenti va utilizzato un apposito canale Slack\texttrademark, mentre per i committenti va utilizzato l'indirizzo \email\space \href{mailto:\GroupEmail}{\GroupEmail}.\\
Il \textit{Responsabile di Progetto} deve occuparsi di chiedere ai proponenti o committenti una riunione quando necessario e, una volta ricevuta una proposta di luogo e data, deve assicurarsi della disponibilità dei membri del gruppo. In caso di impossibilità per il gruppo di svolgere la riunione nella data o luogo proposti, è compito del \textit{Responsabile di Progetto} negoziare un nuovo incontro che permetta il raggiungimento del numero minimo di partecipanti tra i membri del gruppo. Tale numero è cinque.\\
Nel caso di impossibilità di un incontro di persona, si valuterà, discutendone anche con i proponenti o committenti, il mezzo tramite cui incontrarsi virtualmente.

\paragraph{Svolgimento delle riunioni}\mbox{}\\
A differenza di quanto riportato alla \sezione \ref{sec:svolgimento_riunioni_interne} relativa allo svolgimento delle riunioni interne, non è presente la fase preliminare della discussione dei progressi e delle decisioni, ma va trattato direttamente l'ordine del giorno.

\paragraph{Verbale delle riunioni}\mbox{}\\
La verbalizzazione delle riunioni esterne deve avvenire conformemente a quanto riportato nella \sezione \ref{sec:verbale_riunioni_interne}. Il verbale di ciascuna di queste riunioni fa parte della documentazione del progetto.\\
Questo documento deve essere redatto utilizzando un l'apposito template \LaTeX.

\subsection{Repository e strumenti per la condivisione dei file}
\subsubsection{Repository}
Per il versionamento sia dei documenti che del codice sorgente del progetto sono impiegati due repository di Git\texttrademark. Il repository del codice sorgente ha licenza MIT ed è possibile accedervi al seguente indirizzo: \url{<indirizzo da mettere>}.
La documentazione ha lo stesso tipo di licenza e è mantenuta su un altro repository Git\texttrademark al seguente indirizzo: \url{<indirizzo da mettere>}.\\
Il repository dei documenti è suddiviso in due cartelle: 
\begin{itemize}
	\item \textit{Documenti interni}, che contiene tutti i documenti interni;
	\item \textit{Documenti esterni}, che contiene tutti i documenti esterni.
\end{itemize}
Tutti i partecipanti al progetto vanno aggiunti come collaboratori ai due repository.

\subsubsection{Condivisione dei file}
Tutti i file principali del progetto vanno resi disponibili sulla piattaforma GitHub\texttrademark. Per tutti gli altri file di cui sia necessaria la condivisione va utilizzato il servizio offerto da Google Drive\texttrademark.

