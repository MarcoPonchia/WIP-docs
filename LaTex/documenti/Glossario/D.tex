\letteraGlossario{D}
\definizione{Design Pattern}
Concetto che può essere definito come una soluzione progettuale generale ad un problema ricorrente. Si tratta di una descrizione o modello logico da applicare per la risoluzione di un problema che può presentarsi in diverse situazioni durante le fasi di progettazione e sviluppo del software, ancor prima della definizione dell'algoritmo risolutivo della parte computazionale.

\definizione{Debugging}
Attività che consiste nell'individuazione da parte del programmatore della porzione di software affetta da errore (\glossario{bug}) rilevata nei software a seguito dell'utilizzo del programma.

\definizione{Discord}
Discord è un’applicazione di chat vocale online che offre dei vantaggi come la bassa latenza, server gratis per gli utenti e infrastrutture per server dedicati.\\
\url{https://discordapp.com/}

\definizione{Direttore}
Il direttore è il tipo di utente dell’applicazione che sfrutta quest’ultima per controllare e gestire il flusso di ordinazioni, il magazzino e le consegne.

\definizione{Driver}
Componente attiva fittizia per pilotare i test. Controlla l’esecuzione di procedure che non costituiscano il main di un programma.

\definizione{Docker}
Docker è un progetto \glossario{open source} che automatizza lo sviluppo di applicazioni all'interno di container software, fornendo un'astrazione aggiuntiva grazie alla virtualizzazione a livello di sistema operativo di \glossario{Linux}. Docker utilizza le funzionalità di isolamento delle risorse del kernel \glossario{Linux} per consentire a "container" indipendenti di coesistere sulla stessa istanza di \glossario{Linux}, evitando l'installazione e la manutenzione di una macchina virtuale.\\
\url{https://www.docker.com/}

\definizione{DOM}
Document Object Model (DOM), letteralmente modello a oggetti del documento, è una forma di rappresentazione dei documenti strutturati come modello orientato agli oggetti.
DOM è lo standard ufficiale del \glossario{W3C} per la rappresentazione di documenti strutturati in maniera da essere neutrali sia per la lingua che per la piattaforma. DOM è inoltre la base per una vasta gamma di interfacce di programmazione delle applicazioni; alcune di esse sono standardizzate dal \glossario{W3C}.\\
\url{https://www.w3.org/DOM/}
\clearpage