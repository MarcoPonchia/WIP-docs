\letteraGlossario{T}
\definizione{Tag}
Un tag (cioè etichetta, marcatore, identificatore) è una parola chiave o un termine associato a un'informazione (un'immagine, una mappa geografica, un post, un video clip ...), che descrive l'oggetto rendendo possibile la classificazione e la ricerca di informazioni basata su parole chiave.

\definizione{Task}
Un task su \glossario{Teamwork} è un’attività pianificata a cui si possono assegnare risorse.

\definizione{Teamwork}
Tool di \glossario{project management} disponibile online 
Permette di gestire  task, tickets e \glossario{milestones} ed è dotato di strumenti per il tracciamento del tempo di lavoro.

\definizione{Teamwork Project}
Teamwork Project è l’applicazione web per la gestione di progetti inclusa nel pacchetto \glossario{Teamwork}.

\definizione{Telegram}
Servizio di \glossario{instant messaging} multipiattaforma, usato per inviare messaggi tramite Internet.

\definizione{\TeX}
ll rendering è un termine della lingua inglese che in senso esteso indica la resa (o restituzione) grafica, ovvero un'operazione compiuta da un disegnatore per produrre una rappresentazione di qualità di un oggetto o di una architettura (progettata o rilevata).\\
\TeX -rendering indica la renderizzazione che viene fatta tramite il linguaggio strutturato \TeX, solitamente per le espressioni matematiche.

\definizione{TextEdit}
Casella di testo editabile è un \glossario{elemento di input} che fornisce all’utente la possibilità di inserire un input oppure un dato di tipo testuale.

\definizione{TeXMaker}
TeXMaker è un editor gratuito per la creazione di documenti in \glossario{\LaTeX} che offre strumenti per il controllo della sintassi e il visualizzatore integrato.\\
\url{http://www.xm1math.net/texmaker/}

\definizione{TeXstudio}
TeXstudio è un programma gratuito per la creazione di documenti in \glossario{\LaTeX} che offre strumenti per il controllo della sintassi e il visualizzatore integrato.\\
\url{http://www.texstudio.org/}

\definizione{To-do list}
Lista di attività da svolgere nella quale è possibile spuntare ogni singolo compito quando viene portato a termine (attraverso \glossario{elemento grafico} o/e \glossario{elemento funzionale}).

\definizione{Travis}
Travis CI è un servizio \glossario{hosted} e distribuito di integrazione continua usato per costruire e testare progetti software ospitati su \glossario{GitHub}.\\
\url{https://travis-ci.org/}

\definizione{Trello}
Trello è un’applicazione per la gestione di progetto, che offre una versione gratuita con funzioni limitate.\\
\url{https://trello.com/}

\definizione{Twitter Boostrap}
Raccolta di strumenti per la creazione di siti web, contenente modelli basati su \glossario{HTML}, \glossario{CSS} e \glossario{Javascript}.
\clearpage