\input{../../modello/layout}
\input{../../modello/global}
\input{local}

\begin{document}

\makeFrontPage

\input{diario_modifiche}

\clearpage
\tableofcontents

\letteraGlossario{A}
\definizione{API}
Il termine API(acronimo di application programming interface) indica ogni insieme di procedure disponibili al programmatore, spesso organizzate in gruppi di strumenti utili per uno specifico compito.

\definizione{Analisi dei requisiti}
Documento che definisce i requisiti trovati durante la fase di analisi per definire le funzionalit\`a che il nuovo prodotto deve offrire.
\clearpage

\letteraGlossario{B}
\definizione{Bottone}
Interfaccia grafica che permette all’utente di interagire con l’applicazione.L’input varia a seconda della tipologia del bottone. 

\definizione{Bottone Radio}

\definizione{Bubble generica}
Interfaccia principale che rappresenta un contenitore per gli elementi di input, di output e uno strato di logica costruito tramite le funzionalit\`a base offerte dal \glossario{framework}.

\definizione{Bubble memory}
La bubble memory è un oggetto \glossario{javascript} in cui viene salvato lo stato della bolla e di eventuali variabili utilizzate per tenere traccia dello stato delle componenti di input/output.
La durata di questa memoria è uguale alla durata della bolla stessa.
\clearpage

\letteraGlossario{C}
\definizione{Caso d'uso}
Funzionalità di un prodotto software, o tecnica usata nei processi di ingegneria del software per effettuare in maniera esaustiva e non ambigua la raccolta dei requisiti, al fine di produrre software di qualità.

\definizione{Checkbox}
\glossario{Elemento grafico} che permette all’utente di selezionare o rimuovere un determinato \glossario{elemento funzionale}.

\definizione{Collection}
Insieme di dati organizzati in una struttura determinata.In \glossario{MongoDB} i dati vengono organizzati sotto forma di tabelle relazionali.

\definizione{Commit}
Parlando di controllo di versione, un commit si effettua quando si copiano le modifiche fatte su file locali nella cartella del \glossario{repository}.

\definizione{Commitente}
Figura che commissiona il capitolato d’appalto.

\definizione{CSS}
Acronimo di Cascading Style Sheets, ovvero fogli di stile, \`e un linguaggio usato nella formattazione di pagine web.

\definizione{CSS3}
Specifiche riguardati i fogli di stile \glossario{CSS}, costituite da sezioni separate dette moduli, con differenti stati di avanzamento e stabilit\`a. Alcuni di questi moduli sono stati pubblicati formalmente come \glossario{W3C Recommendation}, nel novembre 2014.
\clearpage

\letteraGlossario{D}
\definizione{Design Pattern}
Concetto che può essere definito come una soluzione progettuale generale ad un problema ricorrente. Si tratta di una descrizione o modello logico da applicare per la risoluzione di un problema che può presentarsi in diverse situazioni durante le fasi di progettazione e sviluppo del software, ancor prima della definizione dell'algoritmo risolutivo della parte computazionale.

\definizione{Debugging}
Attività che consiste nell'individuazione da parte del programmatore della porzione di software affetta da errore (bug) rilevata nei software a seguito dell'utilizzo del programma.

\definizione{Documenti esterni}
Documenti la cui lista di distribuzione contenga elementi del team ed elementi esterni al team.

\definizione{Documenti interni}
Documenti la cui lista di distribuzione contiene solo componenti del team.
\clearpage

\letteraGlossario{E}
\definizione{Elemento}
Elemento del \glossario{framework} generico che può essere un \glossario{elemento grafico}, un \glossario{elemento funzionale}, un \glossario{elemento di input}, un \glossario{elemento di output} oppure un insieme di questi.

\definizione{Elemento funzionale}
Elemento del \glossario{framework} con una determinata funzionalità che ha lo scopo realizzare le operazioni esposte solitamente all’utente attraverso l’interfaccia grafica(\glossario{elemento grafico}).

\definizione{Elemento grafico}
Elemento dell’interfaccia grafica del \glossario{framework} con lo scopo di poter visualizzare dei dati o permettere all’utente di interagire.

\definizione{Elemento di input}
\glossario{Elemento grafico} che può essere aggiunto alla \glossario{bubble generica} per svolgerne le funzionalit\`a di input. Ogni elemento di input verr\`a assegnato ad una variabile all’interno della \glossario{bubble memory}. Gli elementi di input possono essere di vari tipi. Per maggiori dettagli sui tipi di input vedere sezione xxxxx del documento yyyyyy.

\definizione{Elemento di output}
\glossario{Elemento grafico} che può essere aggiunto alla \glossario{bubble generica} per svolgerne le funzionalit\`a di output. Ogni elemento di output verr\`a assegnato ad una variabile all’interno della \glossario{bubble memory}. Gli elementi di output possono essere di vari tipi. Per maggiori dettagli sui tipi di output vedere sezione xxxxx del documento yyyyyy.
\clearpage

\letteraGlossario{F}
\definizione{Framework}
Architettura logica di supporto (spesso un'implementazione logica di un particolare \glossario{design pattern}) su cui un software può essere progettato e realizzato, spesso facilitandone lo sviluppo da parte del team di sviluppatori.
\clearpage

\letteraGlossario{G}
\definizione{Gantt}
Ideatore del \textit{diagramma di Gantt}, strumento usato nelle attività di \glossario{project management} per tenere sotto controllo tutte le attività correlate al progetto in una determinata fascia temporale.

\definizione{Git}
Sistema software di controllo di versione distribuito.

\definizione{GitHub}
GitHub è un servizio web di hosting per lo sviluppo di progetti software, che usa il sistema di controllo di versione \glossario{Git}.

\definizione{Google Drive}
Piattaforma che permette di archiviare, condividere, modificare e visualizzare diversi tipi di file.
\clearpage

\letteraGlossario{H}
\definizione{Hosting}
Spazio per l’archiviazione di file.

\definizione{HTML}
Linguaggio di markup per la strutturazione di pagine web.

\definizione{HTML5}
Linguaggio \glossario{HTML} pubblicato come \glossario{W3C Recommendation} da ottobre 2014.

\definizione{HTTP}
Acronimo di HyperText Transfer Protocol \`e un protocollo utilizzato come sistema per la trasmissione di informazioni sul web. Le specifiche del protocollo sono gestite dal World Wide Web Consortium (W3C).

\definizione{HTTPS}
Acronimo di  HyperText Transfer Protocol over Secure Socket Layer \`e l'applicazione della crittografia asimmetrica al protocollo \glossario{HTTP}, per garantire trasferimenti di dati nel web evitando attacchi di tipo \glossario{man in the middle}.
\clearpage

\letteraGlossario{I}
\definizione{IDE}
Acronimo per Integrated Development Environment, ovvero un ambiente di sviluppo integrato per la realizzazione di programmi per sistemi informatici.

\definizione{ISO}
Abbreviazione per International Organization for Standardization, è la più importante organizzazione a livello mondiale per la definizione di norme tecniche.
\clearpage

\letteraGlossario{J}
\definizione{Java}
linguaggio di programmazione orientato agli oggetti, progettato per essere indipendente dalla piattaforma di esecuzione, utilizzando l'implementazione di un processore virtuale detto \glossario{JVM}.

\definizione{Javascript}
Linguaggio di scripting orientato agli oggetti e agli eventi, utilizzato nella programmazione Web lato client per la creazione di effetti dinamici n siti e applicazioni web.

\definizione{JSON}
Acronimo di JavaScript Object Notation, \`e un formato usato nell''interscambio di dati fra applicazioni client-server.

\definizione{JVM}
Acronimo di Java Virtual Machine \`e il componente della piattaforma Java che esegue i programmi tradotti in \glossario{bytecode} dopo una prima compilazione.
\clearpage

\letteraGlossario{L}
\definizione{Label}
Il label è un \glossario{elemento grafico} che mostra delle informazioni o dei dati testuali.
\clearpage

\letteraGlossario{M}
\definizione{Milestone}
Importante traguardo intermedio nello svolgimento del progetto. Molto spesso è rappresentata da eventi, cioè da attività con durata zero o di un giorno, e viene evidenziata in maniera diversa dalle altre attività nell'ambito dei documenti di progetto. Può essere intesa anche come una particolare configurazione di item relativi al progetto.

\definizione{MongoDB}
Database non relazionale classificato come \glossario{NoSQL}, orientato ai documenti si tratta di software libero e \glossario{open source}.
\clearpage

\letteraGlossario{N}
\definizione{Node.js}
\glossario{Framework} per lo sviluppo di applicazioni server-side di Javascript

\definizione{NoSQL}
Acronimo di Not Only SQL rappresentano tutti cui software che non utilizzano il modello relazionale, definendo le basi di date costruite in questo modo come memorizzazioni strutturate.
\clearpage

\letteraGlossario{O}
\definizione{Open source}
Accostato ad un software sta ad indicare che il codice sorgente dello stesso \`e pubblico, favorendone lo studio, le modifiche ed estensioni da parte di programmatori indipendenti.
\clearpage

\letteraGlossario{P}
\definizione{Package}
Collezione di classi e interfacce correlate.

\definizione{PDF}
Il PDF (Portable Document Format) è un formato di file usato per presentare e scambiare documenti in modo affidabile, indipendentemente dal software, dall'hardware o dal sistema operativo.

\definizione{PERT}
Il PERT acronimo per program evaluation and review technique è uno strumento di creazione di diagrammi per la pianificazione di progetto tramite l’analisi delle risorse impiegate e la suddivisione delle attività a calendario.

\definizione{Piano di progetto}
Documento redatto dal Responsabile di progetto e dall’Amministratore al fine di stimare i costi, i tempi e le risorse necessari alla realizzazione del progetto.

\definizione{Piano di qualifica}
Documento il cui scopo è quello di definire gli obiettivi di qualità del prodotto , le strategie e le risorse necessarie alla qualità di processo.

\definizione{PNG}
Sigla per Portable Network Graphics, ovvero un formato file per immagini.

\definizione{Project Management}
Insieme di attività svolte tipicamene da una figura dedicata e specializzata detta project manager, volte all'analisi, alla progettazione, alla pianificazione e alla realizzazione degli obiettivi di un progetto, gestendolo in tutte le sue caratteristiche e fasi evolutive, nel rispetto di precisi vincoli (tempi, costi, risorse, scopi, qualità).

\definizione{Proponenti}
Azienda o ente che propone il capitolato d’appalto.
\clearpage

\letteraGlossario{R}
\definizione{Repository}
Luogo di memorizzazione dei file, spesso situato in un server remoto.

\definizione{Revert}
In ambito di \glossario{controllo di versione}, è l'abbandono di uno o più cambiamenti recenti in favore di un ritorno ad una precedente versione di un documento o di parti di software.

\definizione{RoketChat}
Piattaforma di messaggistica online attraverso browser, applicazione desktop o mobile.
\clearpage

\letteraGlossario{S}
\definizione{Skype}
Software proprietario freeware di \glossario{instant messaging} e VoIP. Con esso sono possibili le videochiamate e lo scambio di messaggi testuali o di file.

\definizione{SQL}
Acronimo di Structured Query Language \`e un linguaggio standardizzato per database che utilizzano il modello relazionale.

\definizione{Studio di fattibilità}
Documento il cui scopo è quello di descrivere le valutazioni effettuate nella scelta del capitolato d’appalto.

\definizione{SVG}
Acronimo di Scalable Vector Graphics, indica una tecnologia in grado di visualizzare oggetti di grafica vettoriale e, pertanto, di gestire immagini scalabili dimensionalmente.
\clearpage

\letteraGlossario{T}
\definizione{Telegram}
Servizio di \glossario{instant messaging} multipiattaforma, usato per inviare messaggi tramite Internet. 

\definizione{Teamwork}
Tool di \glossario{project management} disponibile online 
Permette di gestire  task, tickets e \glossario{milestones} ed è dotato di strumenti per il tracciamento del tempo di lavoro.

\definizione{TextEdit}
Casella di testo editabile è un \glossario{elemento di input} che fornisce all’utente la possibilità di inserire un input oppure un dato di tipo testuale.

\definizione{To-do list}
Lista di attività da svolgere nella quale è possibile spuntare ogni singolo compito quando viene portato a termine (attraverso \glossario{elemento grafico} o/e \glossario{elemento funzionale}).
\clearpage

\letteraGlossario{U}
\definizione{UML}
Acronimo per Unified Modeling Language, è un linguaggio visuale di modellazione e specifica basato sul paradigma object-oriented.

\definizione{Utente cliente}
Il cliente è il tipo di utente dell’applicazione che sfrutta quest’ultima per consultare il menù, selezionare i piatti desiderati ed effettuare l’ordinazione al ristorante.

\definizione{Utente cuoco}
Il cuoco è il tipo di utente dell’applicazione che sfrutta quest’ultima per visualizzare le ordinazione da preparare e notificare quando sono pronte.

\definizione{Utente direttore}
Il direttore è il tipo di utente dell’applicazione che sfrutta quest’ultima per controllare e gestire il flusso di ordinazioni, il magazzino e le consegne.

\definizione{Utente fattorino}
Il fattorino è il tipo di utente dell’applicazione che sfrutta quest’ultima per visualizzare la lista di consegne da effettuare e confermare quando ne completa una.

\definizione{Utente responsabile degli acquisti}
Il responsabile degli acquisti è il tipo di utente dell’applicazione che sfrutta quest’ultima per visualizzare la lista degli acquisti e spuntare quelli effettuati.

\definizione{UTF-8}
UTF-8, ovvero Unicode Transformation Format, 8 bit, è una codifica dei caratteri Unicode in sequenze di lunghezza variabile di byte.
\clearpage

\letteraGlossario{W}
\definizione{Way of Working}
La prassi, il modo di fare regolato da norme che segue un team o un'azienda nella produzione di prodotti software.

\definizione{Workflow}
Schema che descrive l’automatizzazione delle procedure e i processi di lavoro collaborativo.

\definizione{W3C}
Sintesi di World Wide Web Consortium \`e la principale organizzazione per gli standard del World Wide Web.

\definizione{W3C Recommendation}
Standard formalmente dichiarati da parte del \glossario{W3C}.
\clearpage

\end{document}
